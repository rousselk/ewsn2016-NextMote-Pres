\documentclass[10pt,c]{beamer}

\usepackage[T1]{fontenc}
\usepackage[utf8]{inputenc}


%%%%%%%%%%%%%%%%%%%%%%%%%%%%%%%%%%%%%%%%%%%%%%%%%%%%%%%%%%%%%%%%%%%%%%%%%%%%%%%%
%%%                               80 COLONNES                                %%%
%%%%%%%%%%%%%%%%%%%%%%%%%%%%%%%%%%%%%%%%%%%%%%%%%%%%%%%%%%%%%%%%%%%%%%%%%%%%%%%%

%%% THÈME DE LA PRÉSENTATION
\usetheme{JuanLesPins}  %% Warsaw

%%% ÉLÉMENTS DE TITRE
\title{Using Cooja for WSN Simulations: Some New Uses and Limits}
\author{Kévin Roussel \and Ye-Qiong Song \and Olivier Zendra}
\institute{INRIA Nancy Grand-Est~---
           LORIA UMR~7503~--- Université de Lorraine}
\date{EWSN'2016 NextMote workshop\\
      \textit{15 February 2016}}


%%%%%%%%%%%%%%%%%%%%%%%%%%%%%%%%%%%%%%%%%%%%%%%%%%%%%%%%%%%%%%%%%%%%%%%%%%%%%%%%
%%%                               80 COLONNES                                %%%
%%%%%%%%%%%%%%%%%%%%%%%%%%%%%%%%%%%%%%%%%%%%%%%%%%%%%%%%%%%%%%%%%%%%%%%%%%%%%%%%

%%% COMMANDES UTILES AU COURS DU CORPS DE TEXTE

\renewcommand{\emph}[1]{\textbf{\textit{#1}}}
\newcommand{\nom}[1]{\textbf{#1}}

\setbeamertemplate{itemize item}[square]
\setbeamertemplate{itemize subitem}[triangle]
\setbeamertemplate{itemize subsubitem}[circle]


%%%%%%%%%%%%%%%%%%%%%%%%%%%%%%%%%%%%%%%%%%%%%%%%%%%%%%%%%%%%%%%%%%%%%%%%%%%%%%%%
%%%                               80 COLONNES                                %%%
%%%%%%%%%%%%%%%%%%%%%%%%%%%%%%%%%%%%%%%%%%%%%%%%%%%%%%%%%%%%%%%%%%%%%%%%%%%%%%%%

%%% DÉBUT DE LA PRÉSENTATION
\begin{document}

%%% TITRE
\begin{frame}
\titlepage
\end{frame}

%%% SOMMAIRE GÉNÉRAL

\begin{frame}
\frametitle{Contents}
\tableofcontents
\end{frame}

%%%%%%%%%%%%%%%%%%%%%%%%%%%%%%%%%%%%%%%%%%%%%%%%%%%%%%%%%%%%%%%%%%%%%%%%%%%%%

\section{Introduction}

\begin{frame}
\frametitle{Introduction (1/2)}
\begin{block}{To develop and test large, ambitious WSN-based projects}
$\rightarrow$ need of powerful and accurate simulation/emulation tools
\end{block}
\begin{block}{Many such simulation/emulation tools exist}
\begin{itemize}
\item OpenSim (from OpenWSN project)
\item TOSSIM (from TinyOS project)
\item Cooja (from Contiki OS project)
\end{itemize}
\end{block}
\begin{exampleblock}{This work focus on the latter framework: \nom{Cooja}}
$\rightarrow$ (one of) the most used simulation/emulation tool
\end{exampleblock}
\end{frame}

\begin{frame}
\frametitle{Introduction (2/2)}
\begin{block}{The present work focus on the two following subjects}
\begin{enumerate}
\item The ability---\emph{actually tested and used}---to use the Cooja
framework to simulate/emulate systems not related to Contiki OS
\item The inaccuracies in time-related results we discovered while using
Cooja\footnote{Note we always used the Cooja version provided with
Contiki release 2.7} to perform our own simulations to test our own
WSN-related projects
\end{enumerate}
\end{block}
\end{frame}

%%%%%%%%%%%%%%%%%%%%%%%%%%%%%%%%%%%%%%%%%%%%%%%%%%%%%%%%%%%%%%%%%%%%%%%%%%%%%

\section{Cooja and MSPSim}

\begin{frame}
\frametitle{The Cooja framework and its emulators (1/2)}
\begin{block}{The Cooja application}
It's a Java-based application, providing three main features:
\begin{enumerate}
\item a graphical user interface (GUI) fo the simulation work
\item simulation of the radio medium underlying WSN communications
\item an extensible framework, allowing integration of additional
Java-based tools to the application
\end{enumerate}
$\rightarrow$ this latest feature allow Cooja to embed emulators,
and thus to actually \emph{emulate} WSN motes
\end{block}
\end{frame}

\begin{frame}
\frametitle{The Cooja framework and its emulators (2/2)}
\begin{exampleblock}{The embedded emulators}
These are currently two:
\begin{enumerate}
\item \nom{Avrora}, for emulation of Atmel AVR-based devices
\item \nom{MSPSim}, for emulation of TI MSP430-based devices
\end{enumerate}
$\rightarrow$ \nom{MSPSim} is currently the most used, since
MSP430-based motes are more commonly used. \emph{The present work
focuses on Cooja and the MSPSim emulator only.}
\end{exampleblock}
\end{frame}

%%%%%%%%%%%%%%%%%%%%%%%%%%%%%%%%%%%%%%%%%%%%%%%%%%%%%%%%%%%%%%%%%%%%%%%%%%%%%

\section{Using Cooja and MSPSim: Not Only for Contiki!}

\begin{frame}
\frametitle{Using Cooja and MSPSim with any WSN system\ldots \\
            (or even without)}
\begin{block}{What do Cooja's embedded emulators run?}
\small
The Contiki build system produces standard (ELF format) executables \\
$\rightarrow$ the Cooja embedded emulators are designed to run such
standard executables
\end{block}
\begin{exampleblock}{What does it mean?}
\small
\emph{Any system producing such standard ELF executables (besides Contiki
OS itself) will run on virtual motes emulated with the Cooja framework}

We tested this with MSPSim, using:
\begin{itemize}
\item applications based on RIOT OS
\item ``bare-metal'' applications
\end{itemize}

Any other OS using \texttt{msp430-gcc} as its compiler (e.g.: TinyOS)
should run fine on Cooja/MSPSim: a simple trick on executables' file
extension is enough
\end{exampleblock}
\end{frame}

%%%%%%%%%%%%%%%%%%%%%%%%%%%%%%%%%%%%%%%%%%%%%%%%%%%%%%%%%%%%%%%%%%%%%%%%%%%%%

\section{Timing Inaccuracy Problem in MSPSim}

\begin{frame}
\frametitle{Title}
Une diapositive \LaTeX.
\end{frame}

%%%%%%%%%%%%%%%%%%%%%%%%%%%%%%%%%%%%%%%%%%%%%%%%%%%%%%%%%%%%%%%%%%%%%%%%%%%%%

\section{Consequences}

\begin{frame}
\frametitle{Title}
Une diapositive \LaTeX.
\end{frame}

%%%%%%%%%%%%%%%%%%%%%%%%%%%%%%%%%%%%%%%%%%%%%%%%%%%%%%%%%%%%%%%%%%%%%%%%%%%%%

\section{Conclusion}

\begin{frame}
\frametitle{Title}
Une diapositive \LaTeX.
\end{frame}

%%%%%%%%%%%%%%%%%%%%%%%%%%%%%%%%%%%%%%%%%%%%%%%%%%%%%%%%%%%%%%%%%%%%%%%%%%%%%

\appendix

%%% ANNEXE : REMERCIEMENTS ET RÉFÉRENCES BIBLIOGRAPHIQUES

\section{Appendix}

\subsection{Acknowledgements}

\begin{frame}
\frametitle{Acknowledgements}
\begin{columns}[c]
\begin{column}{8.25cm}
\begin{block}{Funding}
This work has been funded by the French national PIA\footnote{\textsl{PIA~:
\guillemotleft~Programme d'investissement d'avenir~\guillemotright}}
\textbf{LAR} (\textit{``Living Assistant Robot''}).
\end{block}
\end{column}
\end{columns}
\end{frame}

\subsection{References}

\begin{frame}
\frametitle{References}
Une diapositive \LaTeX.
\end{frame}

%%%%%%%%%%%%%%%%%%%%%%%%%%%%%%%%%%%%%%%%%%%%%%%%%%%%%%%%%%%%%%%%%%%%%%%%%%%%%%%%
%%%                               80 COLONNES                                %%%
%%%%%%%%%%%%%%%%%%%%%%%%%%%%%%%%%%%%%%%%%%%%%%%%%%%%%%%%%%%%%%%%%%%%%%%%%%%%%%%%

%%% FIN DE LA PRÉSENTATION
\end{document}



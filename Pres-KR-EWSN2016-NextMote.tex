\documentclass[10pt,c]{beamer}

\usepackage[T1]{fontenc}
\usepackage[utf8]{inputenc}
\usepackage{graphicx}
\usepackage{multirow}


%%%%%%%%%%%%%%%%%%%%%%%%%%%%%%%%%%%%%%%%%%%%%%%%%%%%%%%%%%%%%%%%%%%%%%%%%%%%%%%%
%%%                               80 COLONNES                                %%%
%%%%%%%%%%%%%%%%%%%%%%%%%%%%%%%%%%%%%%%%%%%%%%%%%%%%%%%%%%%%%%%%%%%%%%%%%%%%%%%%


%%% THÈME DE LA PRÉSENTATION
\usetheme{JuanLesPins}  %% Warsaw

%%% ÉLÉMENTS DE TITRE
\title{Using Cooja for WSN Simulations: Some New Uses and Limits}
\author{Kévin Roussel \and Ye-Qiong Song \and Olivier Zendra}
\institute{INRIA Nancy Grand-Est~---
           LORIA UMR~7503~--- Université de Lorraine}
\date{EWSN'2016 NextMote workshop\\
      \textit{15 February 2016}}


%%%%%%%%%%%%%%%%%%%%%%%%%%%%%%%%%%%%%%%%%%%%%%%%%%%%%%%%%%%%%%%%%%%%%%%%%%%%%%%%
%%%                               80 COLONNES                                %%%
%%%%%%%%%%%%%%%%%%%%%%%%%%%%%%%%%%%%%%%%%%%%%%%%%%%%%%%%%%%%%%%%%%%%%%%%%%%%%%%%

%%% COMMANDES UTILES AU COURS DU CORPS DE TEXTE

\renewcommand{\emph}[1]{\textbf{\textit{#1}}}
\newcommand{\nom}[1]{\textbf{#1}}
\newcommand{\tblcaption}[1]{\textbf{\textsl{\small#1}}\vspace{0.1cm}}

\setbeamertemplate{itemize item}[square]
\setbeamertemplate{itemize subitem}[triangle]
\setbeamertemplate{itemize subsubitem}[circle]
\setbeamertemplate{bibliography item}{\insertbiblabel}
% Ajout des numéros de transparents à la barre de navigation en bas à droite
\setbeamercolor{footline}{fg=darkgray}
\setbeamerfont{footline}{series=\bfseries,size=\scriptsize}
\addtobeamertemplate{navigation symbols}{}{%
    \usebeamerfont{footline}%
    \usebeamercolor[fg]{footline}%
    \hspace{1.5em}%
    \insertframenumber~/ 17%\inserttotalframenumber
}


%%%%%%%%%%%%%%%%%%%%%%%%%%%%%%%%%%%%%%%%%%%%%%%%%%%%%%%%%%%%%%%%%%%%%%%%%%%%%%%%
%%%                               80 COLONNES                                %%%
%%%%%%%%%%%%%%%%%%%%%%%%%%%%%%%%%%%%%%%%%%%%%%%%%%%%%%%%%%%%%%%%%%%%%%%%%%%%%%%%

%%% DÉBUT DE LA PRÉSENTATION
\begin{document}

%%% TITRE
\begin{frame}
\titlepage
\end{frame}

%%% SOMMAIRE GÉNÉRAL

\begin{frame}
\frametitle{Contents}
\tableofcontents
\end{frame}

%%%%%%%%%%%%%%%%%%%%%%%%%%%%%%%%%%%%%%%%%%%%%%%%%%%%%%%%%%%%%%%%%%%%%%%%%%%%%

\section{Introduction}

\begin{frame}
\frametitle{Introduction I}
\begin{block}{To develop and test large, ambitious WSN-based projects}
$\rightarrow$ need of powerful and accurate simulation/emulation tools
\end{block}
\begin{block}{Many such simulation/emulation tools for WSNs exist:}
\begin{itemize}
\item OpenSim (from OpenWSN project)
\item TOSSIM (from TinyOS project)
\item Cooja (from Contiki OS project)
\end{itemize}
\end{block}
\begin{exampleblock}{This work focus on the latter framework: \nom{Cooja}}
$\rightarrow$ (one of) the most used WSN simulation/emulation tool
\end{exampleblock}
\end{frame}

\begin{frame}
\frametitle{Introduction II}
\begin{block}{The present work focuses on the following contributions:}
\begin{enumerate}
\item The ability---\emph{actually tested and used}---to use the Cooja
framework to simulate/emulate systems not related to Contiki OS
\item The inaccuracies in time-related results we discovered while using
Cooja\footnote{Note we always used the Cooja version provided with
Contiki release 2.7} to perform our own simulations to test our own
WSN-related projects
\item the possible consequences of these inaccuracies in the literature
published in the WSN domain
\end{enumerate}
\end{block}
\end{frame}

%%%%%%%%%%%%%%%%%%%%%%%%%%%%%%%%%%%%%%%%%%%%%%%%%%%%%%%%%%%%%%%%%%%%%%%%%%%%%

\section{Cooja and MSPSim}

\begin{frame}
\vspace{-0.5cm}
\frametitle{The Cooja framework and its emulators}
\center{\includegraphics[width=10cm]{CoojaFramework.png}}

\vspace{0.25cm}
\begin{block}{}
The present work focuses on Cooja and the MSPSim emulator only
\end{block}
\end{frame}

%%%%%%%%%%%%%%%%%%%%%%%%%%%%%%%%%%%%%%%%%%%%%%%%%%%%%%%%%%%%%%%%%%%%%%%%%%%%%

\section{Using Cooja and MSPSim: Not Only for Contiki!}

\begin{frame}
\frametitle{Using Cooja and MSPSim with any WSN system\ldots \\
            (or even without)}
\begin{block}{What do Cooja's embedded emulators run?}
\small
The Contiki build system produces standard (ELF format) executables \\
$\rightarrow$ the Cooja embedded emulators are designed to run such
standard executables
\end{block}
\begin{exampleblock}{What does it mean?}
\small
\emph{Any system producing such standard ELF executables (besides Contiki
OS itself) will run on virtual motes emulated with the Cooja framework}

We tested this with MSPSim, using:
\begin{itemize}
\item applications based on RIOT OS
\item ``bare-metal'' applications
\end{itemize}

Any other OS using \texttt{msp430-gcc} as its compiler (e.g.: TinyOS)
should run fine on Cooja/MSPSim: a simple trick on executables' file
extension is enough
\end{exampleblock}
\end{frame}

%%%%%%%%%%%%%%%%%%%%%%%%%%%%%%%%%%%%%%%%%%%%%%%%%%%%%%%%%%%%%%%%%%%%%%%%%%%%%

\section{Timing Inaccuracy Problem in MSPSim}

\subsection{The Setup}

\begin{frame}
\frametitle{Timing Inaccuracy Problem in MSPSim: Setup}
\begin{alertblock}{What did we discover during our experiments?}
\begin{itemize}
\item unexplained delays during simulations/emulations of packet
transmissions (TX), not observed on real hardware
\item the differences appear on one peculiar operation: \emph{when loading
the TX buffer of the CC2420 radio transceiver}
%(common to both our hardware platforms: Sky/TelosB and Zolertia Z1 motes)
\end{itemize}
\end{alertblock}
\begin{block}{What did we do to investigate that problem?}
\begin{itemize}
\item we wrote a simple test program, which only sends data packets of
various sizes (resp. 30, 60 and 110 bytes of MAC payload)
%to which a MAC layer-related overhead of 11 bytes is always added
\item we tried this program on various configurations:
  \begin{itemize}
  \item two different hardware platforms: Sky/TelosB and Zolertia Z1 motes
  \item two different WSN OS: Contiki and RIOT OS
  \end{itemize}
\end{itemize}
\end{block}
\end{frame}

\subsection{The Results}

\begin{frame}
\frametitle{Timing Inaccuracy Problem in MSPSim: Results I}
\begin{center}
\vspace{-0.9cm}
\small
\[
\mbox{\textbf{Inaccuracy}}
   = \frac{\mbox{Simulated delay} - \mbox{Actual Hardware test delay}}
          {\mbox{Actual Hardware test delay}}
\]

\vspace{0.1cm}

\tblcaption{\small Observed TX buffer loading delay inaccuracies of MSPSim
            emulation \\ versus tests on actual hardware}
\begin{tabular}{|l|l|r|r|}
\hline
\textbf{HW platform} & \textbf{OS} & \textbf{Payload size}
                     & \textbf{Inaccuracy} \\
\hline
SkyMote/TelosB & Contiki &  30 bytes &  11\% \\
SkyMote/TelosB & Contiki &  60 bytes &  15\% \\
SkyMote/TelosB & Contiki & 110 bytes &  13\% \\
\hline
SkyMote/TelosB & RIOT OS &  30 bytes &  15\% \\
SkyMote/TelosB & RIOT OS &  60 bytes &  16\% \\
SkyMote/TelosB & RIOT OS & 110 bytes &  18\% \\
\hline
Zolertia Z1    & Contiki &  30 bytes & 122\% \\
Zolertia Z1    & Contiki &  60 bytes & 114\% \\
Zolertia Z1    & Contiki & 110 bytes &  95\% \\
\hline
Zolertia Z1    & RIOT OS &  30 bytes & 184\% \\
Zolertia Z1    & RIOT OS &  60 bytes & 185\% \\
Zolertia Z1    & RIOT OS & 110 bytes & 181\% \\
\hline
\end{tabular}
\end{center}
\end{frame}

\begin{frame}
\frametitle{Timing Inaccuracy Problem in MSPSim: Results I}
\begin{block}{What did our tests show?}
\begin{itemize}
\item the inaccuracy problem is mainly influenced by the emulated hardware
platform
  \begin{itemize}
  \item \emph{on Zolertia Z1 motes, timing inaccuracies are catastrophic:
  delays are overestimated by amounts from 95\% to 185\%}
  \item on Sky/TelosB motes, timing inaccuracies are much less important:
  delays are under- or overestimated by amounts up to 18\%
  \end{itemize}
\item the used OS has much less influence on the problem: results tend
to be more accurate on Contiki than on RIOT OS
\item the size of transmitted packets doesn't seem to have any
significative influence on this problem
\end{itemize}
\end{block}
\end{frame}

\begin{frame}
\frametitle{Timing Inaccuracy Problem in MSPSim: Results II}
\begin{center}
\vspace{-0.4cm}
\includegraphics[width=8cm]{Delays.png}
\vspace{0.1cm}

\tblcaption{\small Relative weight of TX buffer loading
            in packet transmission delays}
\small
\begin{tabular}{|l|l|r|r|}
\hline
\multirow{2}{2.5cm}{\textbf{HW platform}}
 & \multirow{2}{1cm}{\textbf{OS}}
  & \multirow{2}{2cm}{\textbf{Payload size}}
     & \multicolumn{1}{|c|}{\textbf{\underline{Loading delay}}} \\
 & & & \multicolumn{1}{|c|}{\textbf{Total delay}} \\
\hline
SkyMote/TelosB & Contiki &  30 bytes & 13\% \\
SkyMote/TelosB & Contiki &  60 bytes & 13\% \\
SkyMote/TelosB & Contiki & 110 bytes & 12\% \\
\hline
SkyMote/TelosB & RIOT OS &  30 bytes & 57\% \\
SkyMote/TelosB & RIOT OS &  60 bytes & 53\% \\
SkyMote/TelosB & RIOT OS & 110 bytes & 51\% \\
\hline
Zolertia Z1    & Contiki &  30 bytes & 10\% \\
Zolertia Z1    & Contiki &  60 bytes & 11\% \\
Zolertia Z1    & Contiki & 110 bytes & 10\% \\
\hline
Zolertia Z1    & RIOT OS &  30 bytes & 52\% \\
Zolertia Z1    & RIOT OS &  60 bytes & 48\% \\
Zolertia Z1    & RIOT OS & 110 bytes & 46\% \\
\hline
\end{tabular}
\end{center}
\end{frame}

\begin{frame}
\frametitle{Timing Inaccuracy Problem in MSPSim: Results II}
\begin{block}{What did our tests show \textit{(bis)}?}
\emph{The TX buffer loading operation weight in the whole packet
transmission time is anything but negligible:}

This TX buffer loading operation amounts to about half of the total
packet transmission delay under RIOT OS (much less on Contiki,
where it represents 10\% to 13\%) \\
$\rightarrow$ \textit{This is probably due to the way these OSes do manage
the SPI bus that links the MCU to the radio transceiver on a mote
(``fast mode'' for Contiki OS, ``safe mode'' for RIOT OS)}
\end{block}
\end{frame}

\subsection{Discussion}

\begin{frame}
\frametitle{Timing Inaccuracy Problem in MSPSim: Discussion}
\begin{alertblock}{What can we deduce from these results?}
\begin{itemize}
\item \emph{Cooja/MSPSim simulations/emulations do not provide accurate
enough timing-related results}, especially on Zolertia Z1 motes \\
$\rightarrow$ with this inaccuracy issue, \emph{the simulations/emulations
made with this framework cannot be used for timing evaluation purposes
of WSN-based projects}
\item since the hardware seems to be the main factor influencing this
problem, we suppose its cause is---at least partially---linked to wrong
timing calibrations for microcontrollers' emulation \\
%{\small (Sky/TelosB microcontroller emulation being visibly much more
%accurate than Zolertia Z1's)}
\end{itemize}
\end{alertblock}
\end{frame}

%%%%%%%%%%%%%%%%%%%%%%%%%%%%%%%%%%%%%%%%%%%%%%%%%%%%%%%%%%%%%%%%%%%%%%%%%%%%%

\section{Consequences}

\begin{frame}
\vspace{-0.25cm}
\frametitle{Consequences on WSN Literature}
\begin{block}{What did we find?}
\begin{itemize}
\item many recent publications rely on Cooja/MSPSim to perform, directly
or indirectly, time-related performance evaluations of their WSN-based
projects
\cite{Constrain-Routing-Trees-2014} \cite{DINAS-2014}
\cite{Efficient-Distrib-Svc-Discovery-2014} \cite{IETF-Routing-WSN-2014}
\cite{TinySDN-2014} \cite{Trickle-L2-2014}
\cite{Visual-Sensor-Networks-2014} \cite{Key-Mgmt-2015}
%(fortunately, most of them use the relatively less impacted Sky/TelosB
%hardware platform)
\item most of these publications are focused on higher layers of WSN
network stacks, like routing protocols 
\cite{Constrain-Routing-Trees-2014}
\cite{IETF-Routing-WSN-2014} \cite{Trickle-L2-2014}
or application-level projects
\cite{DINAS-2014} \cite{Efficient-Distrib-Svc-Discovery-2014}
\cite{Visual-Sensor-Networks-2014} \cite{Key-Mgmt-2015} \\
$\rightarrow$ \emph{their authors may probably be unaware of this problem,}
since they are not focused on low-level details
\end{itemize}
\end{block}
\vspace{-0.25cm}
\begin{alertblock}{What does that mean?}
Many WSN studies rely (at least partially) on Cooja/MSPSim to evaluate
their time-related performances \\
$\rightarrow$ \emph{This inaccuracy problem could make their results
unreliable, and thus put their conclusions in jeopardy!}
\end{alertblock}
\end{frame}

%%%%%%%%%%%%%%%%%%%%%%%%%%%%%%%%%%%%%%%%%%%%%%%%%%%%%%%%%%%%%%%%%%%%%%%%%%%%%

\section{Conclusions}

\begin{frame}
\frametitle{Conclusions I}
\begin{block}{Our contributions}
\begin{enumerate}
\item we showed that the Cooja framework is not limited to simulations
of systems running Contiki OS, but can run any program designed for the
emulated architectures (especially MSP430, thanks to MSPSim)
\item we showed that the MSPSim emulation---at least for the version
provided with Contiki release 2.7---suffers from a serious timing inaccuracy
issue; we described the extent of the problem, and provided serious clues
about its cause(s)
\item we briefly enumerated a (non-exhaustive) list of recent publications,
showing the magnitude of the negative impact this problem can cause
\end{enumerate}
\end{block}
\end{frame}

\begin{frame}
\frametitle{Conclusions II}
\begin{alertblock}{In summary:}
Until a fix is made and published to correct this issue in MSPSim,
\emph{we think that all works based on Cooja/MSPSim simulations to evaluate
WSN projects---especially for timing matters---should be checked (even
partially) by tests on actual hardware}
\end{alertblock}
\begin{exampleblock}{How to test the accuracy of your own works?}
We provided the source of our test programs (for Contiki and RIOT OS)
on a public GitHub repository:
\center{\texttt{https://github.com/rousselk/tim-inacc-tst-prg}}
\end{exampleblock}
\end{frame}

\begin{frame}
\frametitle{Conclusions III}
\begin{exampleblock}{Strengths of Cooja/MSPSim}
\begin{itemize}
\item the many publications in the domain of WSNs, especially very recent
ones, including Cooja/MSPSim simulations, is a testimony of the usefulness
of such an emulation framework (especially to test large WSNs, which is
often difficult and expensive to with actual hardware) \\
$\rightarrow$ \emph{That's why we strongly believe fixing Cooja/MSPSim
is really important, and hope for such a fix to appear soon}
\item while the issue described can impair the use of Cooja/MSPSim as
a performance evaluation tool, \emph{it does not affect its other
valuable uses, like the ability to develop and debug WSN-related
software much more easily thanks to its emulation features}
\end{itemize}
\end{exampleblock}
\end{frame}

%%%%%%%%%%%%%%%%%%%%%%%%%%%%%%%%%%%%%%%%%%%%%%%%%%%%%%%%%%%%%%%%%%%%%%%%%%%%%

\appendix

%%% ANNEXE : REMERCIEMENTS ET RÉFÉRENCES BIBLIOGRAPHIQUES

\section{Appendix}

\begin{frame}[b]
\frametitle{Thanks for Your Attention}
\begin{block}{}
\center{\textbf{\Large Any questions?}}
\end{block}
\vspace{1.25cm}
\begin{columns}[c]
\begin{column}{8cm}
\begin{block}{\small Acknowledgements (Funding)}
\small
This work has been funded by the French national PIA
\footnote{\scriptsize\textsl{PIA~: \guillemotleft~Programme
d'investissement d'avenir~\guillemotright}}
\textbf{LAR} (\textit{``Living Assistant Robot''}).
\end{block}
\end{column}
\end{columns}
\vspace{0.3cm}
\end{frame}

\begin{frame}[t]
\frametitle{References}
\tiny
\bibliographystyle{unsrt}
\bibliography{TimingInacc}
\end{frame}


%%%%%%%%%%%%%%%%%%%%%%%%%%%%%%%%%%%%%%%%%%%%%%%%%%%%%%%%%%%%%%%%%%%%%%%%%%%%%%%%
%%%                               80 COLONNES                                %%%
%%%%%%%%%%%%%%%%%%%%%%%%%%%%%%%%%%%%%%%%%%%%%%%%%%%%%%%%%%%%%%%%%%%%%%%%%%%%%%%%


%%% FIN DE LA PRÉSENTATION
\end{document}



